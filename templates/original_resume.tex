\documentclass[10pt]{article}

\usepackage{titlesec}
\usepackage{titling}
\usepackage[margin=.5in]{geometry}
\usepackage{bold-extra}
\usepackage{enumitem}% http://ctan.org/pkg/enumitem
\usepackage{multicol} % in the preamble
\usepackage{tabularx}
\usepackage{array}
\setlist[itemize]{topsep=0pt,itemsep=0pt,parsep=0pt,partopsep=0pt}

% =================================
% Formatting various headings 
% and their spacings
% =================================
\titleformat{\section}
            {\Large\scshape} % Formatting of the text
            {}  % Before the text
            {0em} % Spacing before the text
            {}  % After the section
            [\titlerule] % horizontal line
\titleformat{\subsection}
            {\bfseries\large} % Formatting of the text
            {}  % Before the text
            {0em} % Spacing before the text
            {}  % After the section
\titleformat{\subsubsection} %runin makes it so it spaces on the same line
            {\bfseries} % Formatting of the text
            {}  % Before the text
            {0em} % Spacing before the text
            {}  % After the section
            []   % adds a little space
            
\titlespacing{\section}
    {0em}
    {.2em}
    {.4em}
    
\titlespacing{\subsection}
    {0em}
    {0em}
    {0em}            
\titlespacing{\subsubsection}
    {0em}  % Left Margin
    {.25em}  % Space before 
    {0em}  % Space afterwards
% =====================================================

% ==========================
% Creates the subtitle for
% positions where you can
% put intern, etc.
% ==========================
\newcommand{\positiontitle}[1]
        {
            \hspace{-1.8em}
            \textit{#1}
        }

% ====================
% Generates the full
% formatting for adding
% a position to your 
% resume
% ====================
\newcommand{\makeposition}[4]
    {
        \subsection{#1 \hfill {\normalsize #3}}
        \positiontitle{#2 \hfill {\normalsize #4}}    
    }


% ============================
% Generates the education layout
% ============================

    
\newcommand{\contentitem}
    {
        \item\small
    }
    
\newcommand{\onelinecontent}[2]
    {
        \small\textbf{#1} & #2
    }
    
\newcommand{\skac}[4]
{
    \begin{flushleft}
     \begin{tabularx}{\linewidth}{
        >{\hsize=.525\hsize}X% 10% of 4\hsize 
        >{\hsize=1.475\hsize}X% 30% of 4\hsize
        >{\hsize=.4\hsize}X% 10% of 4\hsize 
        >{\hsize=1.6\hsize}X% 30% of 4\hsize
           % sum=4.0\hsize for 4 columns
      }
        #1  &   #2 \\
      \end{tabularx}
     \begin{tabularx}{\linewidth}{
        >{\hsize=.25\hsize}X% 10% of 4\hsize 
        >{\hsize=1.75\hsize}X% 30% of 4\hsize
           % sum=4.0\hsize for 4 columns
      }
        #3  \\
        #4  \\
      \end{tabularx}
       \end{flushleft}

}

% ===================================
% Resume Header Area 
% ===================================
\renewcommand{\maketitle} %changing the command maketitle
        {
            \begin{center}
                {\huge\bfseries\theauthor}\\
                \vspace{.25em}
                \VAR{meta.email} $\bullet$ \VAR{meta.phone} $\bullet$ \VAR{meta.address}
            \end{center}
            \vspace{-.7em}
            \titlerule[2pt]
        }

% ===========================================


\begin{document}
    \title{Resume}
    \author{\VAR{meta.name}}
    \maketitle
    \thispagestyle{empty}

    \BLOCK{ for section in content }
        %% set current_section = content[section].content
        \section{\VAR{section}}
        \BLOCK{ if content[section].get("type") == None }
            \BLOCK{ for entry in current_section}
                %% set current_entry = current_section[entry]
                \makeposition{\VAR{entry}}{\VAR{current_entry.description}}{\VAR{current_entry.location}}{\VAR{current_entry.date}}
                \begin{itemize}
                    \BLOCK{ for content in current_entry.content }
                        \contentitem \VAR{tex_escape(content)}
                    \BLOCK{ endfor }
                \end{itemize}
            \BLOCK{ endfor }
        \BLOCK{ elif content[section].type == "one-line-section" }
            %% set current_section = template_code.parse_one_line_section(current_section)
            \begin{flushleft}
                \BLOCK{for entry in current_section}
                    %% set current_entry = current_section[entry]
                    \BLOCK{ if current_entry["type"] == "half" }
                        \begin{tabularx}{\linewidth}{
                        >{\hsize=.525\hsize}X% 10% of 4\hsize 
                        >{\hsize=1.475\hsize}X% 30% of 4\hsize
                        >{\hsize=.4\hsize}X% 10% of 4\hsize 
                        >{\hsize=1.6\hsize}X% 30% of 4\hsize
                            % sum=4.0\hsize for 4 columns
                        }
                        \onelinecontent{\VAR{tex_escape(entry)}}{\VAR{tex_escape(current_entry["content"])}}  &   \onelinecontent{\VAR{tex_escape(current_entry["title2"])}}{\VAR{tex_escape(current_entry["content2"])}} \\
                        \end{tabularx}
                    \BLOCK{ elif current_entry["type"] == "full" }
                        \begin{tabularx}{\linewidth}{
                            >{\hsize=.25\hsize}X% 10% of 4\hsize 
                            >{\hsize=1.75\hsize}X% 30% of 4\hsize
                                % sum=4.0\hsize for 4 columns
                            }
                            \onelinecontent{\VAR{tex_escape(entry)}}{\VAR{tex_escape(current_entry["content"])}} \\
                        \end{tabularx}
                    \BLOCK{ endif }
                \BLOCK{ endfor }


            \end{flushleft}
        \BLOCK{ elif content[section].type == "education" }
            \BLOCK{ for college in current_section }
                \subsubsection{\VAR{college} \hfill {\normalsize \VAR{current_section[college].location}}}
                \subsubsection{\VAR{current_section[college].college} \hfill {\normalsize \textnormal{Expected Graduation: \VAR{current_section[college].graduation}}}}
                \noindent\ignorespaces
                \textit{\VAR{current_section[college].degree[0]} \hfill {\normalsize \textnormal{GPA: \VAR{current_section[college].gpa}}}}
                \BLOCK{ for degree in current_section[college].degree[1:]}
                    \newline
                    \textit{\VAR{degree}}
                \BLOCK{ endfor }
            \BLOCK{ endfor }
        \BLOCK{ endif }
    \BLOCK{ endfor }
\end{document}

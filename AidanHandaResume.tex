\documentclass[10pt]{article}

\usepackage{titlesec}
\usepackage{titling}
\usepackage[margin=.5in]{geometry}
\usepackage{bold-extra}
\usepackage{enumitem}% http://ctan.org/pkg/enumitem
\usepackage{multicol} % in the preamble
\usepackage{tabularx}
\usepackage{array}
\usepackage{tikz,tikzpagenodes}
\usetikzlibrary{calc}
\usepackage{refcount}
\setlist[itemize]{topsep=0pt,itemsep=0pt,parsep=0pt,partopsep=0pt}

% =============================
% Creating a lined item section
% =============================

\newcounter{linedlist} % new counter for amount of lists
\newcounter{linedcount}[linedlist] % create new item counter
\newcounter{linedtmp}[linedlist] % tmp counter needed for checking before/after current item

\newcommand{\drawoptionsconn}{gray, shorten <= .5mm, shorten >= .5mm, thick}
\newcommand{\drawoptionsshort}{gray, shorten <= .5mm, shorten >= -1mm, thick}

\newcommand{\lineditem}{% Modified `\item` to update counter and save nodes
  \small
  \stepcounter{linedcount}%
  \item[\linkedlist{%
    i\alph{linedlist}\arabic{linedcount}}]%
  \label{item-\alph{linedlist}\arabic{linedcount}}%
  \ifnum\value{linedcount}>1%
    \ifnum\getpagerefnumber{item-\alph{linedlist}\arabic{linedtmp}}<\getpagerefnumber{item-\alph{linedlist}\arabic{linedcount}}%
      \begin{tikzpicture}[remember picture,overlay]%
        \expandafter\draw\expandafter[\drawoptionsshort] (i\alph{linedlist}\arabic{linedcount}) --
          ++(0,3mm) --
          (i\alph{linedlist}\arabic{linedcount} |- current page text area.north);% draw short line
      \end{tikzpicture}%
    \else%
      \begin{tikzpicture}[remember picture,overlay]%
        \expandafter\draw\expandafter[\drawoptionsconn] (i\alph{linedlist}\arabic{linedtmp}) -- (i\alph{linedlist}\arabic{linedcount});% draw the connecting lines
      \end{tikzpicture}%
    \fi%
  \fi%
  \addtocounter{linedtmp}{2}%
  \IfRefUndefinedExpandable{item-\alph{linedlist}\arabic{linedtmp}}{}{% defined
    \ifnum\getpagerefnumber{item-\alph{linedlist}\arabic{linedtmp}}>\getpagerefnumber{item-\alph{linedlist}\arabic{linedcount}}%
      \begin{tikzpicture}[remember picture,overlay]%
      \expandafter\draw\expandafter[\drawoptionsshort] (i\alph{linedlist}\arabic{linedcount}) --
        ++(0,-3mm) --
        (i\alph{linedlist}\arabic{linedcount} |- current page text area.south);% draw short line
      \end{tikzpicture}%
    \fi%
  }%
  \addtocounter{linedtmp}{-1}%
}

\newcommand{\linkedlist}[1]{
  \raisebox{0pt}[0pt][0pt]{\begin{tikzpicture}[remember picture]%
  \node (#1) [gray,circle,fill,inner sep=1.5pt]{};
  \end{tikzpicture}}%
}

\newenvironment{lineditemize}{%
% Create new `lineditemize` environment to keep track of the counters
  \stepcounter{linedlist}% increment list counter
  \begin{itemize}[leftmargin=*]
}{\end{itemize}%
  }

% =================================
% Formatting various headings 
% and their spacings
% =================================
\titleformat{\section}
            {\Large\scshape} % Formatting of the text
            {}  % Before the text
            {0em} % Spacing before the text
            {}  % After the section
            [\titlerule] % horizontal line
\titleformat{\subsection}
            {\bfseries\large} % Formatting of the text
            {}  % Before the text
            {0em} % Spacing before the text
            {}  % After the section
\titleformat{\subsubsection} %runin makes it so it spaces on the same line
            {\bfseries} % Formatting of the text
            {}  % Before the text
            {0em} % Spacing before the text
            {}  % After the section
            []   % adds a little space
            
\titlespacing{\section}
    {0em}
    {.2em}
    {.4em}
    
\titlespacing{\subsection}
    {0em}
    {0em}
    {0em}            
\titlespacing{\subsubsection}
    {0em}  % Left Margin
    {.25em}  % Space before 
    {0em}  % Space afterwards
% =====================================================

% ==========================
% Creates the subtitle for
% positions where you can
% put intern, etc.
% ==========================
\newcommand{\positiontitle}[1]
        {
            \hspace{-1.8em}
            \textit{#1}
        }

% ====================
% Generates the full
% formatting for adding
% a position to your 
% resume
% ====================
\newcommand{\makepositionheader}[2]
    {
        \subsection{#1 \hfill {\normalsize #2}}
    }
\newcommand{\makepositiondesc}[2]
    {
        \positiontitle{#1 \hfill {\normalsize #2}}    
    }


% ============================
% Generates the education layout
% ============================

    
\newcommand{\contentitem}
    {
        \item\small
    }
    
\newcommand{\onelinecontent}[2]
    {
        \small\textbf{#1} & #2
    }
    
\newcommand{\skac}[4]
{
    \begin{flushleft}
     \begin{tabularx}{\linewidth}{
        >{\hsize=.525\hsize}X% 10% of 4\hsize 
        >{\hsize=1.475\hsize}X% 30% of 4\hsize
        >{\hsize=.4\hsize}X% 10% of 4\hsize 
        >{\hsize=1.6\hsize}X% 30% of 4\hsize
           % sum=4.0\hsize for 4 columns
      }
        #1  &   #2 \\
      \end{tabularx}
     \begin{tabularx}{\linewidth}{
        >{\hsize=.25\hsize}X% 10% of 4\hsize 
        >{\hsize=1.75\hsize}X% 30% of 4\hsize
           % sum=4.0\hsize for 4 columns
      }
        #3  \\
        #4  \\
      \end{tabularx}
       \end{flushleft}

}

% ===================================
% Resume Header Area 
% ===================================
\renewcommand{\maketitle} %changing the command maketitle
        {
            \begin{center}
                {\huge\bfseries\theauthor}\\
                \vspace{.25em}
                ahanda@gatech.edu $\bullet$ (630) 363-4860 $\bullet$ 685 Chippewa Dr. Naperville, IL 60563
            \end{center}
            \vspace{-.7em}
            \titlerule[2pt]
        }

% ===========================================


\begin{document}
    \title{Resume}
    \author{Aidan Handa}
    \maketitle
    \thispagestyle{empty}

            \section{Education}
                                    \subsubsection{Georgia Institute of Technology \hfill {\normalsize Atlanta, GA}}
                \subsubsection{College of Computing \hfill {\normalsize \textnormal{Expected Graduation: December 2021}}}
                \noindent\ignorespaces
                \textit{Bachelor of Science in Computer Science \hfill {\normalsize \textnormal{GPA: 4.0}}}
                                                \section{Professional Experience}
                                    \makepositionheader{DreamHire.io}{Remote - San Francisco, CA}
                                    \makepositiondesc{Software Engineering Intern}{September 2018 - August 2019}
                    \begin{itemize}
                                                    \contentitem Created a modular PDF annotation engine enabling highlighting, drawing, etc., using Typescript
                                                    \contentitem Developed user interfaces within the login and onboarding process using HTML/CSS/Javascript
                                                    \contentitem Engineered an efficient matching algorithm in C\# to pair recruiters and candidates based on interests
                                            \end{itemize}
                                                \section{Projects}
                                    \makepositionheader{Xplicate}{Naperville, IL}
                                    \makepositiondesc{A data backed AI agent designed to summarize complex medical research papers}{May 2020}
                    \begin{itemize}
                                                    \contentitem Experimented with machine learning and algorithmic language representations including tf-idf and word2vec
                                                    \contentitem Built in Python and Node.js using serverless architecture deployed through Google Cloud Functions
                                                    \contentitem Utilized various APIs including Google Cloud ML, Google Cloud NLP, and the Wikipedia Python API
                                                    \contentitem Delivered a working viable product under a strict 24 hour time constraint
                                            \end{itemize}
                                            \makepositionheader{Isolation AI}{Atlanta, GA}
                                    \makepositiondesc{An intelligent agent trained genetically to play the game Isolation}{December 2019}
                    \begin{itemize}
                                                    \contentitem Modeled the problem mathematically and programmatically using object-oriented techniques in Python
                                                    \contentitem Assessed a variety of network structures on effectiveness and ability to develop specific intuition
                                                    \contentitem Evaluated the effectiveness of genetically training the network by teaching it to play against itself
                                                    \contentitem Implemented extensive and compute expensive training of the model using C++ and Boost
                                            \end{itemize}
                                            \makepositionheader{Lilac}{Atlanta, GA}
                                    \makepositiondesc{An artifically intelligent personal shopper built over a span of 36 hours at HackGT.}{November 2019}
                    \begin{itemize}
                                                    \contentitem Architected core project functionalities, specifications, and system design
                                                    \contentitem Developed a concurrent scraping utility for dynamically rendered websites in Python
                                                    \contentitem Deployed serverless code on GCP infrastructure to facilitate data for Google Dialogflow
                                                    \contentitem Evaluated pitfalls with critical features and patched problems under time pressure
                                            \end{itemize}
                                            \makepositionheader{Go Music-Bot}{Naperville, IL}
                                    \makepositiondesc{An online collaborative music server application for use in group settings.}{January 2019 - March 2019}
                    \begin{itemize}
                                                    \contentitem Assembled a modularized and containerized real-time music control interface using Docker and Go
                                                    \contentitem Designed the app to run asynchronously using multiprocessing for concurrent connections
                                                    \contentitem Programmed a YouTube scraper in Python and Go to pull music and feed the app’s queue
                                            \end{itemize}
                                                \section{Leadership Experience}
                                    \makepositionheader{Urban Smart Shoes}{Naperville, IL}
                                    \makepositiondesc{Lead Designer}{May 2019 - July 2019}
                    \begin{itemize}
                                                    \contentitem Organized and managed a team of three rising college freshman throughout the innovation process
                                                    \contentitem Built a cross-platform mobile app as a Bluetooth remote control and data-hub using Flutter and Dart
                                                    \contentitem Crafted a 3D CAD model of the module case using Autodesk Fusion360
                                                    \contentitem Advised and tutored the embedded programming of the shoes in C++
                                            \end{itemize}
                                            \makepositionheader{App-I-Tite}{Naperville, IL}
                                    \makepositiondesc{Lead Software Engineer}{January 2018 - May 2018}
                    \begin{itemize}
                                                    \contentitem Led a team of 4 software development students using Agile methodologies
                                                    \contentitem Constructed and maintained critical features using Swift and MongoDB for data storage
                                                    \contentitem Devised payment flow for credit and debit cards using the Stripe API
                                                    \contentitem Implemented user authentication using Node.JS and Parse backend
                                                    \contentitem Shaped the user interface and user experience of the mobile app using Sketch and Interface Builder
                                            \end{itemize}
                                                \section{Skills And Achievements}
                    \begin{flushleft}
                                                            \begin{tabularx}{\linewidth}{
                        >{\hsize=.525\hsize}X% 10% of 4\hsize 
                        >{\hsize=1.475\hsize}X% 30% of 4\hsize
                        >{\hsize=.4\hsize}X% 10% of 4\hsize 
                        >{\hsize=1.6\hsize}X% 30% of 4\hsize
                            % sum=4.0\hsize for 4 columns
                        }
                        \onelinecontent{Programming}{Python, Java, Javascript/Typescript, C}  &   \onelinecontent{Distinctions}{Faculty Honors, Dean's List} \\
                        \end{tabularx}
                                                                                \begin{tabularx}{\linewidth}{
                            >{\hsize=.25\hsize}X% 10% of 4\hsize 
                            >{\hsize=1.75\hsize}X% 30% of 4\hsize
                                % sum=4.0\hsize for 4 columns
                            }
                            \onelinecontent{Technologies}{AWS (Lambda, DynamoDB, API Gateway), GCP (Functions, ML, NLP), Docker, Numpy, Pandas} \\
                        \end{tabularx}
                                                                                \begin{tabularx}{\linewidth}{
                            >{\hsize=.25\hsize}X% 10% of 4\hsize 
                            >{\hsize=1.75\hsize}X% 30% of 4\hsize
                                % sum=4.0\hsize for 4 columns
                            }
                            \onelinecontent{Coursework}{Data Structures \& Algorithms, Object-Oriented Design, Introduction to Artifical Intelligence, Linear Algebra, Discrete Mathematics} \\
                        \end{tabularx}
                                    

            \end{flushleft}
            \end{document}
